\documentclass[11]{article}

\usepackage{graphicx}
\usepackage{url}
\usepackage{float}
\usepackage{caption}
\usepackage[margin = 1in]{geometry}
\usepackage{indentfirst}
\usepackage{natbib}
\usepackage{subcaption}
\usepackage{hyperref}

%underlines the title of a book/article in the References section (ideal for UWS Harvard style)
%FIXME check that all the citation rules are obeyed.
\usepackage{ulem}

\begin{document}

\begin{titlepage}
	\begin{center}	
		\includegraphics[width = 5cm,height = 1.5cm]{uws_logo.png}\\[5cm]
	
{ \huge \bfseries %
		ASP.NET vs PHP\\ \Large
		 \textit{Which technology to choose for the next version of uws.ac.uk?} \\ 
}
	\vspace{2cm}			
			
		\begin{flushright}
				\large Student:\\
				Marius-Lucian Olariu\\[1cm]
		\end{flushright}
		
	
		\begin{flushleft}
			 \large
				Senior Lecturer: \\
				Graeme McRobbie \\[1cm]
		\end{flushleft}
		
	\vspace{2cm}	
	
		
		\vfill
		
		{\large {Paisley \\ 2019}}
		\end{center}
\end{titlepage}

\newpage

\tableofcontents

\listoffigures

\listoftables

\newpage

\section{Introduction}

\bibpunct{(}{)}{;}{a}{,}{}

	Nowadays, people expect personalized experiences and the Internet makes no exception to this rule. In order to fulfill this user requirement we need dynamic webpages (i.e. pages that can change automatically the content they display), few webpages currently are static (i.e. have the same content every time you open them). An example of a dynamic webpage could be \textit{amazon.com} that displays on the home page products that you might be interested in (based on your previous purchases, search history) whereas an example of a static webpage could be the contents of a book published online (e.g. https://natureofcode.com/book/). Developers automatically wonder what are the best technologies to use in order to develop dynamic webpages. However, due to the fact that different dynamic webpages have different requirements this study-case is going to focus on comparing two web technologies suitable for developing a future version of \textit{uws.ac.uk}, namely ASP.NET and PHP. One of the main requirements of this site is to be able to process a lot of data (e.g. a lot of applications for programmes provided by the university) thus it uses a SQL database system to store it. \\ 
	
	ASP.NET is a framework from Microsoft for building dynamic web pages that was released in 2002 as a successor to \textbf{A}ctive \textbf{S}erver \textbf{P}ages (ASP).  This technology is used on the server-side of web development; in other words, it interacts with the "computer" processing the website request and produces web pages that contain only HTML, CSS and JavaScript. ASP.NET provides flexibility to the developers because they have multiple choices when it comes to what languages to use for the backend of the website, namely ASP.NET supports C\#, Visual Basic .NET and J\# (a language inspired by Java). Additionally, writing the backend in C\# or J\# speeds the process of providing the website to the client because these languages are compiled (often compilation for a webpage backend code happens only once and then the compiled webpage is cached for further use). In 2014 Microsoft made the .NET server stack development open-source (ASP.NET, .NET compiler, .NET Core Runtime etc.), ASP.NET can be found on a GitHub repository (\textit{github.com/aspnet}). Moreover, in the same year they started to support cross-platform development by making the tools needed available for platforms like Linux or macOS. Also, the deployment of a webpage developed using ASP.NET is now possible on a Linux web server  through~\cite{Mono} which is an open-source project and recently Microsoft started sponsoring it.\\

	PHP is one of the most used programming languages \citep{cass2014top, cass20152015} and it is mainly used to write the backend of a webpage, in other words,  a server-side language. PHP was released in 1994 and the acronym stands for \textit{Personal Home Page Tools}. The main advantage of this language is the fact that it can be embedded in HTML code and after being interpreted the PHP code results is HTML code (that gets sent to the browser). It is worth mentioning that PHP development is  open-source, has a huge community around it. Websites developed using PHP for the backend are cross-platform, namely, they can be hosted on web servers running Linux or Windows. The wide adoption of this language is due to the fact that is free to develop with it (no need to pay for an IDE licence) and runs fast in combination with Apache, MySQL and Linux operating system (LAMP stack) on a low-specifications web server~\citep{converse2004php5}. Another reason for the use of PHP is that it is easy to learn.

\pagebreak

\section{Comparison}
%The body must present a comparison between ASP.NET and PHP web sites when used with typical SQL-based database management systems.
%In particular, you should present what you consider to be the five most important factors when comparing these two technologies.

	Firstly, the above technologies are analysed based on how they interact with \textbf{R}elational \textbf{D}atabase \textbf{M}anagement \textbf{S}ystems (RDBMS). In order to manipulate the data stored in a RDBMS the standard language is \textbf{S}tructured \textbf{Q}uery \textbf{L}anguage (SQL) that is why these RDBMS are sometimes referred as \textit{SQL-databases}. There are many RDBMS systems out there, to mention but a few: Oracle, MySQL, Microsoft SQL Server, PostgreSQL etc. . PHP websites use mostly MySQL because these two technologies have been developed with each other in mind, are both open source projects with active web communities around them and (as mentioned above) are fast~\citep{davis2007learning}. ASP.NET websites use mostly Microsoft SQL Server from a practical reason, namely both of them integrate really well being developed by Microsoft and at a time ASP.NET could not be paired with other type of database - the alternative to LAMP stack was WISA stack (Windows OS, Internet Information Services, Microsoft SQL Server, ASP.NET).

%Given the above mentioned things about the bounds between the technologies, the first thing to analyze is how  these two technologies interract with MySql and SQL Server.\\ 

%	\subsection{MySQL}
%	First, let's have a look on how PHP is used with MySQL. PHP5 provides native APIs in order to connect to MySQL, namely \textit{mysql}, {mysqli}  and \textit{PDO\_MySQL}. However, starting with PHP7 the \textit{mysql} extension is deprecated.

%	Second, let's have a look on how ASP.NET is used with MS SQL Server. \\

%	Can you use PHP and MS SQL Server? What about ASP.NET and MySQL.

	\subsection{Storage}
		In the past database storage was quite expensive but as the hardware technology advances and new discoveries are put in practice, for example (if cloud storage is chosen) having data centers at the bottom of oceans~\citep{simon2018project}, the price of storage is becoming cheaper. Nonetheless, there are few things that have to be taken into account when choosing a database system. The considerations will be briefly mentioned and taking them into account the technologies will be compared.

	\subsubsection{Scalability\\}
	As the university keeps expanding, for example - opens new campuses, it is likely that a type of SQL-database cannot keep the pace with the new requirements for \textit{uws.ac.uk} (e.g. store/retrieve more data) thus it would be wise to pick a database system that offers  easy migration of data to another database management system if needed.

	\subsubsection{Database replication\\}
	No matter how scalable a database is at some point the database needs to be replicated in more than one server. This creates the need for keeping the servers synchronized, thus there is created a \textit{master-slave} relationship between the database servers. Usually, the inserts happen in the master database and the reads are supplied by the slave databases (they keep synched with the master database by periodically updating their content). Moreover, in case the master database fails one of the slave databases could take the place to ensure the availability of the website.

	\subsubsection{GUI Interface \\}
	Often in the development process there is a need to interact directly with the database so a GUI can be really helpful rather than just a shell to see the output, whether a web interface or a native software for a specific OS.\\

	PHP offers native support for the following types of databases (see Table~\ref{phpDbSupp}).

		\begin{table}[H]
			\caption{Main databases supported by PHP through native APIs~\citep{PhpDbs}}
			\label{phpDbSupp}
			  \centering
				\begin{tabular}{|l|l|l|}
					\hline
					\textbf{Name} & \textbf{Type} & \textbf{Language}\\
					\hline
					Cubrid & Relational & SQL\\
					\hline
					DB++ & Relational & AQl\\
					\hline
					dBase & Relational & dBase\\
					\hline
					InterBase & Relational & SQL\\
					\hline 
					FrontBase & Relational & SQL\\
					\hline 
					IBM Db2 & Relational & SQL\\
					\hline 
					IBM Informix & Relational & SQL\\
					\hline 
					Ingres & Relational & QUEL\\
					\hline 
					MaxDb & Relational & SQL\\
					\hline 
					MongoDB & Document-oriented database & Mongo Shell\\
					\hline 
					mSQL & Relational & SQL\\
					\hline 
					Microsoft SQL Server & Relational & SQL\\
					\hline 
					MySQL& Relational & SQL\\
					\hline 
					Oracle  & Relational & SQL\\
					\hline 
					PostgreSQL  & Object-Relational & SQL\\
					\hline 
					Paradox  & Relational & PAL\\
					\hline 
					Ingres & Relational & QUEL\\
					\hline 
				\end{tabular}
		\end{table}

		ASP.NET developers can work with a database using \textbf{E}ntity \textbf{F}ramework \textbf{C}ore (EFC), a object-relational mapping that uses .NET objects to manipulate data. ASP.NET supports the following databases - Table~\ref{aspDbSupp}.

		\begin{table}[H]
			\caption{Main databases supported by ASP.NET through EFC~\citep{AspDbs}}
			\label{aspDbSupp}
			  \centering
				\begin{tabular}{|l|l|l|}
					\hline
					\textbf{Name} & \textbf{Type} & \textbf{Language}\\
					\hline
					MariaDB & Relational & SQL\\
					\hline
					MyCAT Server& Relational & SQl\\
					\hline
					Firebird & Relational & SQL\\
					\hline 
					Progress OpenEdge & Relational & OpenEdge ABL\\
					\hline 
					IBM Db2 & Relational & SQL\\
					\hline 
					IBM Informix & Relational & SQL\\
					\hline 
					Ingres & Relational & QUEL\\
					\hline 
					Microsoft SQL Server & Relational & SQL\\
					\hline 
					MySQL& Relational & SQL\\
					\hline 
					Oracle  & Relational & SQL\\
					\hline 
					PostgreSQL  & Object-Relational & SQL\\
					\hline 
				\end{tabular}

		\end{table}


	As can be seen above, for both technologies there is a large variety of databases available. Some databases are available for both technologies like Oracle, PostgreSql, MySQL, Microsoft SQL Server, Ingres, IBM Informix, IBM Db2. There can be said that PHP offers support for almost every database that ASP.NET supports and some more. Therefore, if one is looking for database scalability (section 2.1.1) it is better to develop the backend in PHP.\\
	\indent
	When it comes to Database Replication (section 2.1.2) it seems that MySQL, PostgreSQL and Oracle offer support; Oracle being the best from this perspective. Since both technologies can use the three aforementioned databases we cannot say which one is better. 
	\indent
	When it comes to GUI Interfaces (see 2.1.2.) most of them provide at least a web GUI. One could look further to see which type of database provides the best GUI or the largest number of GUIs (e.g. web, native application etc.) but this is beyond the purpose of this work. Again, it seems that PHP is more likely to be the right choice.\\
	\indent
	Considering the facts presented above about storage/databases it seems that PHP is the right choice.
%talk about how php and asp.net connect to different dbs and how these satisfy the aforementioned useful db characteristics

%Windows OS on server is not free
%Sql server isn't free as well
	

	\subsection{Cost}
	The wide adoption of PHP happened due to the fact that it was open-source and able to used in combination with free software like Linux as web server operating system, Apache and MySQL (the LAMP stack). By contrast, in order to develop dynamic webpages with ASP.NET you had to buy a licence for the IDE, develop on a machine running a licensed Windows, pay for Microsoft SQL Server licence, host your webpage on a licensed Windows OS for the  web server and so on, however, some of that changed when the development for ASP.NET went open-source, Visual Studio free (and available on macOS as well) and the deployment of webpage on a Linux web server was possible through the \cite{Mono}. This was a big and important step for Microsoft. Moreover, now the ASP.NET supports connection to free databases like MySQL. From what was stated above it can be said that the costs for both technologies can be pretty similar if the right combination is chosen.\\
	\indent
	A future version of a webpage like \textit{uws.ac.uk} would need a team of developers in order to be developed professionally and according to \textit{payscale.com} the average salary for a PHP developer is less than the average salary for an ASP.NET developer. A developer working with the WISA stack is, in general, better paid because he can prove through~\cite{MScert} that he has the knowledge necessary for developing a dynamic web app. Therefore, if not for the cost of a certain stack of technologies used then at least from the perspective of workforce cost PHP is cheaper.

		\begin{figure}[H]
				\centering
						\begin{subfigure}{.5\textwidth}
						  \centering
						  \includegraphics[scale=0.45]{aspSalary}
						  \caption{Average salary for a ASP.NET developer in UK}
						  \label{fig:sub1}
						\end{subfigure}%

						\begin{subfigure}{.5\textwidth}
						  \centering
						  \includegraphics[scale=0.45]{phpSalary}
						  \caption{Average salary for a PHP developer in UK}
						  \label{fig:sub2}
						\end{subfigure}

				\caption{Comparison between salaries of PHP/ASP.NET developers}
				\label{salaryComp}
		\end{figure}
	
	% \subsection{Security}

	\subsection{Support}
		The LAMP stack  is open-source thus it is very attractive for a developer/small team of developers to build a small-size dynamic web application. There is a vibrant community around the LAMP stack, thus a large pool of developers available. However, if there is a bug in a certain library function, for example, the approach to get support by using a technology or another is different. For PHP, one could post a forum question and the "friendly" community would provide a workaround. In contrast, for ASP.NET there can be submitted a bug report and the Microsoft paid developers of ASP.NET would then have to fix the problem and include it in the next release of ASP.NET. Now since both platforms are open-source the difference between the two is in the fact that ASP.NET has software architects  paid to oversee the further development of the platform whereas the PHP development is done by developers around the world in their free time.\\
	\indent
	Next, if the university is using other software products from Microsoft (e.g. Microsoft Outlook mail) it would be slightly easier to integrate them with a future version of \textit{uws.ac.uk} developed using ASP.NET. The downside of WISA stack is that the technologies provided are not as customizable as those on the LAMP stack.\\
	
	\indent
	It is known that big enterprises need support for the technologies that they use because they have a big number of employees and often software does not work as it is supposed to. This being said, because Microsoft can provide support for the technologies used in WISA stack a future version of \textit{uws.ac.uk} would be safer to be developed in ASP.NET than PHP.\\

	\subsection{Development}
	The technologies can be compared also by the underlying programming paradigm that they use. In the case of ASP.NET, because it is a framework, C\# will be the chosen language for comparison since it is the most popular choice when developing ASP.NET webpages. \\
	
	\indent
	PHP started as a scripting language in 1995, however, it has moved towards the Object Oriented paradigm supporting concepts like class, object, constructors \& destructors or object inheritance~\citep{PhpOop}. For a language to be considered Object-Oriented it needs to adhere to three main concepts: \textbf{Encapsulation}, \textbf{Polymorphism} and \textbf{Inheritance}. Encapsulation is used to bundle data and methods to a class, to protect the state of an object and it is achieved in PHP through the access specifiers like \textit{public}, \textit{private} and \textit{protected}. Polymorphism is the ability to provide a interface to entities of different types and it is achieved in PHP by having \textit{interfaces} and \textit{abstract classes}. Inheritance allows new objects to "inherit" the properties and services of existing objects and is achieved in PHP by allowing classes to be \textit{extended} and interfaces to be \textit{implemented}. To sum it up, PHP7 can be considered an Object Oriented language but there some problems due to the fact that it has to be compatible with legacy code and the fact that it was not designed from the start as an Object Oriented language.\\
	By comparison, C\# is a robust and mature Object Oriented language and it allows the programmer to develop good designed software that promotes separation of content, logic and data and thus the software it is easier to support in the long run. In addition, ASP.NET provides well-organized libraries that allow the developer to create new features easily  whereas in PHP it might be slightly harder to develop the same functionality by using libraries developed by different developers. \\

	\indent
	A big part of the development time is spent on debugging the code and  testing it. PHP does not offer great tools for debugging, the developer has to use third-party software. ASP.NET provides free of charge an Integrated Development Environment (IDE) for computers running Windows/macOS which allows debugging of webpages code. However, the ASP.NET developers are pretty tied up to the Visual Studio to develop dynamic webpages whereas PHP developers use text editors like Vi/Vim and other open-source technologies that give them greater flexibility. By using small open-source software solutions PHP developers have a thorough understanding of how everything works and they are more capable to integrate new technologies with old ones.\\
	
	\indent
	In the first versions of PHP there was no mechanism for error-trapping but this drawback was fixed in PHP5 when \textit{Exception Handling} was added. Thus, from this way of looking at it, PHP5 is similar to C\#.\\
	
	\indent
	Taking into account strictly the development part it seems that ASP.NET would be a better choice for \textit{uws.ac.uk}.  


	\subsection{Speed}
%alexa.com info
%sumarise the webpage load speed
   According to \textit{similartech.com}, a website which performs comparisons between technologies, ASP.NET was used in the development of about 2.5 million websites whereas PHP was used for the development of about 7.5 million websites~\citep{CompPhpAsp} (see Figure~\ref{websites}). These numbers suggest that both of them are valid technologies used by industry. One way to compare the two technologies would be by the webpage load speed. As can be found on Alexa's website, a company owned by Amazon, the current version of \textit{uws.ac.uk} loads pretty slow, on average 3.64 seconds and 83\% of websites load faster~\citep{LoadSpeed}.
It is generally believed that  ASP.NET is faster than PHP because it can use for the backend programming languages that are compiled, like C\#, whereas PHP is an interpreted language. \\

	\begin{figure}[H]

			\begin{center}
					\includegraphics[scale = 0.3]{websites.png}
			\end{center}
			%the 
			\caption{Comparison of the technologies based on number of websites developed using them}
			\label{websites}
	\end{figure}

	\indent
	There has been found  a study~\citep{mirzoev2014webpage} that compares the technologies by the webpage load speed. The authors created two small types of websites using both technologies: i) test.aspx \& test.php and ii) about.aspx \& about.php. The first type of website, \textit{test.*}, called 3 JavaScript files, 1 style sheet, 4 images and 1 favicon. The second type of website, \textit{about.*}, read a text file with 10.000 lines (each line is 15 characters long) and displayed it in the webpage. The webpage load speed was recorded using an addon, Lori (Life-of-request info), in the Mozilla Firefox browser, the results can be seen in Figure~\ref{speedFig}. In both sessions PHP ran faster than ASP.NET (session 1 -> 0.189 s, session 2 -> 1.281), however, PHP also had one the slowest loading time in session 1 -> 0.304 seconds. In \textit{Session 1} ASP.NET average load time was faster with 11.2 milliseconds than PHP while in \textit{Session 2} PHP website loaded faster on average with 13.56 milliseconds than the one using in ASP.NET. It seems from this study that there is not a big difference between ASP.NET and PHP when it comes to webpage load speed. However, the author believes that a more thorough testing scenario could be designed in order to validate the results of the above-mentioned study. Therefore, it cannot be said what technology is really faster. 

\begin{figure}[H]

	\begin{center}
			\includegraphics[scale = 0.5]{speedComp.png}
	\end{center}
	%the 
	\caption{Speed comparisons between two webpages written using both technologies}
	\label{speedFig}

\end{figure}

%Using php can sometime increase the dev time because of spending much time getting separate components written by other people work together whereas ASP.NET modules generally all work well together 

\pagebreak

\section{Conclusion}
	
	Before Microsoft made ASP.NET "public" there was no doubt which backend development technology to choose for the wide public of developers, namely PHP - the LAMP stack. Developers preferred PHP due to several reasons. Firstly,  some had a business idea (e.g. a startup company) to put in practice and there was a need for a website but they were not sure if the business idea is going to work so the risks were bigger by using ASP.NET than PHP because there was the additional cost (IDE, MS SQL database, IIS licence, Windows OS desktop/server licence). Secondly, a lot of developers had a bias against using technology developed by a big corporation like Microsoft (this is true even today). Last but not least, PHP has and had a huge community around it that was ready to help at any time. By contrast, ASP.NET had a big corporation behind that offered off-the-shelf software and support for it thus it was very appealing for big enterprises. Nowadays, ASP.NET is cheaper to use than in the past and thus the number of websites developed using ASP.NET might increase; we can say that ASP.NET is now appealing to startups or freelance developers. Both technologies have been used to develop well-known webpages (see Table~\ref{websitesTech}). \\ 

		\begin{table}[H]
			\caption{Popular websites developed using PHP or ASP.NET}
			\label{websitesTech}
			  \centering
				\begin{tabular}{|l|l|l|l|}
					\hline
					\textbf{Domain} & \textbf{Launch date} & \textbf{Backend technology} & \textbf{Note}\\
					\hline
					google.com  & 2004 & PHP \& others & \\
					\hline 
					yahoo.com & 1995 & PHP  & \\
					\hline 
					wikipedia.org & 2001 & PHP, Hack  & Hack - dialect of PHP \\
					\hline 
					wordpress.com & 2003 & PHP & \\
					\hline 
					stackoverflow.com & 2008 & ASP.NET  & C\# \\
					\hline 
					w3schools.com & 1998 & ASP.NET  & initially developed using ASP  \\
					\hline 
					microsoft.com & 2005 & ASP.NET  & year when website was published  \\
					\hline 
					dell.com & 2005 & ASP.NET  & \\
					\hline 
				\end{tabular}

		\end{table}
\indent
	Coming back to the main purpose of this essay, making a technology recommendation for the next version of \textit{uws.ac.uk} we have to analyze the outcome of the 5 criteria that have been used for comparison, namely \textit{Storage, Cost, Development,} and \textit{Speed}. From a general point of view, PHP is better when it comes to \textit{cost} and \textit{storage}, whereas ASP.NET is better when it comes to \textit{support} (offered by Microsoft) and \textit{development}; regarding the speed of webpage load both seem to be equally fast. Now taking into account that we are talking about the website of a big university and the reputation is important the cost of webpage development is not such a big concern. Storage is very important but ASP.NET is not far behind PHP when it comes to databases available as can be seen from Table~\ref{phpDbSupp} and Table~\ref{aspDbSupp}. On the other hand, the criteria where ASP.NET is superior, support, is very important for \textit{uws.ac.uk} since it does not afford any downtime. The development in ASP.NET seems to lead to a more maintainable software than PHP. To sum it up, in my opinion ASP.NET is a better choice for the development of the next version of \textit{uws.ac.uk} given the above results of this essay. 

\pagebreak

\bibliographystyle{agsm}
\bibliography{asp_net_vs_php}

\end{document}

